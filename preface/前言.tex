\chapter{前言}

“为材料计算而生”,是抱着多大的觉悟说出这种话啊。这只是一本书,有办法背负其他人的人生吗?

\begin{center}
真是满脑子只想着自己呢……\footnote{以上内容改编自动漫《BanG Dream! It's MyGO!!!!!》中丰川祥子的台词}
\end{center}

俗话说得好,\textit{好记性不如烂笔头},这句话在任何时候都显得格外贴切。尤其是在科研领域(特别是材料科学这样规范和流程化的学科),记录的重要性更是不言而喻。随着计算任务的不断增加,我们掌握的计算方法和参数也日益繁多。将这些知识、操作和方法记录下来,不仅能够帮助我们避免遗忘,还能在遇到类似问题时快速查找,而不必在海量的网络搜索结果中苦苦寻觅。

正是基于这样的考虑,我们结合自己科研团队在材料计算方面遇到的一些实际问题,整理编写了这本书。我们编写这本书的目的有两个:一是为了方便自己,在面对类似问题时能够迅速回忆起解决方案;二是为了通过集体的智慧,汇聚大家的方法和思路,以便在遇到新问题时能够迅速找到答案。

这本书的诞生,既是为了服务于材料计算,也是因材料计算而生。既然如此,我们为何不称它为“为材料计算而生”呢?

我们衷心希望这本书能够惠及更多的人,无论是我们团队的新成员,还是其他团队的老师或同学,都能从中获得帮助。

同时,我们也清楚地认识到自己的能力和知识是有限的,书中的内容难免会有疏漏或错误。我们诚挚地希望读者在使用过程中能够提出宝贵的意见和建议,或者分享你们的经验,共同促进我们的成长和进步。

最后,再次感谢您阅读并使用这本书。

\rightline{Jiaqi Z.}
\rightline{2024年8月 青岛}
\newpage

\section*{如何联系作者}
可通过以下任意一种方式联系:
\begin{itemize}
    \item GitHub的Issue, 这是最直接的方式\footnote{GitHub仓库地址:https://github.com/JackyZhang00/Computational-Materials-Tutorial};
    \item email, 请发送邮件至zhangjq00@sdust.edu.cn或zhangjq\_sd@163.com
\end{itemize}

\section*{如何使用这本书}

在使用时,请按照如下方法:

\begin{enumerate}
    \item 根据研究问题,寻找合适的章节;如果没有,可以在GitHub上提交Issue或者贡献Pull Request;
    \item 在每一节开始,会介绍本节的内容和知识点,查看是否与你的研究问题符合;如果不符合,返回第1步重新查找新的章节;
    \item 阅读这一节内容,并试着针对自己的问题进行操作(或简单检查自己操作是否正确)。如果报错或出现异常结果,进行第4步;如果成功,进行第5步;
    \item 在该节后面的“错误处理”部分,会介绍如何处理报错或异常结果,并给出解决方案。请查找是否有你需要的解决方案,并尝试解决。如果已经解决,进入第5步;否则重新查找新的解决方案;若所有解决方案都无法解决,请提交Issue或者贡献Pull Request;
    \item 放下教程,继续你的研究;或者阅读这一章其他内容,了解其他相关内容。
\end{enumerate}

上述步骤可能(也一定)会重复许多次

\section*{关于本笔记的版权使用说明}

\begin{itemize}
    \item 本书可\emph{免费用于学习, 科研等非商业活动};
    \item 可以以\emph{非商业目的进行传播}, 但在传播过程中\emph{必须保证内容的完整性}(截止到最新发布时, 包括但不限于仓库内Latex源码, pdf文件等. 下同), 需\emph{保证作者信息完整}, 不得进行修改;
    \item 本书\emph{不可用于任何商业用途}(如确有需要, 需联系作者);
    \item 除在GitHub仓库以pull request形式进行编辑修改外, \emph{不允许进行修改并公开传播私自修改版本}(以GitHub仓库版本为标准版本);
    \item 本书著作权归作者(Jiaqi Z.)所有, 其他进行创作的人员也可获得著作权, 其他著作权所有者不得违反上述版权说明;
    \item \emph{如因违反上述说明传播而造成不良影响, 与作者和其他创作者无关}, 特此声明;
    \item 以上说明解释权归Jiaqi Z.所有, 且如有后续更新, 以GitHub仓库最新版说明为准.
\end{itemize}

\section*{创作者名单}

感谢以下人员参与贡献了内容:

Jiaqi Z.