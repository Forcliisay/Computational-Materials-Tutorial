% 请在下方的大括号相应位置填写正确的节标题和标签,以及作者姓名
\section{文件操作}\label{sec:文件操作}
\sectionAuthor{Jiaqi Z.}

% 请在下方的item内填写本节知识点
\begin{Abstract}
    \item 如何移动文件(目录),如何给文件(目录)重命名
    \item 如何删除文件(目录)
    \item 如何复制文件(目录)
\end{Abstract}

% 请在正文相应位置填写正确的小节标题(或小小节标题),同时将标签的“节标题”和“小节标题”改为实际内容

这一节,我们专注于文件的相关操作。类似于Windows的基本操作,Linux的文件操作也无外乎就是\emph{移动、删除、复制}。同时,这一节的许多命令对于文件和目录都是适用的,但可能会有一个注意事项,这往往会出错。

\subsection{移动文件\keyword{mv}}\label{subsec:文件操作-移动文件}

在Linux当中,移动文件使用的命令是\code{mv}。其基本用法是\code{mv <源文件路径> <目标文件路径>}。例如,我们在\code{vasp}目录下,希望将里面的\code{OUTCAR}移动至上一级目录,可以使用\code{mv OUTCAR ..}。类似地,对于更复杂的文件移动,只不过在描述路径时稍微复杂一点,其他的步骤没有什么不同。

如果你足够敏感,也许会发现一点问题:\emph{为什么前面的命令,\code{OUTCAR}是文件,而\code{..}是目录}?两个难道不应该统一吗?

对于这个问题,可以分两个部分讨论:如果前面是文件,后面也是文件,例如\code{mv OUTCAR ../OUTCAR},这个命令与前面的命令效果是完全等价的。但是,有趣的地方在于,如果你试着执行\code{mv OUTCAR ../INCAR}的话,你会发现,Linux将\code{OUTCAR}移动到\code{..}的同时,还将其改名为\code{INCAR}。

进一步想一下,如果我们现在直接写成\code{mv OUTCAR INCAR}的话,可以将其看作把当前目录下的\code{OUTCAR}移动至当前目录,同时改名为\code{INCAR},总的效果就是,文件被重命名为\code{INCAR}。

\begin{attention}
    正如你所见到的那样,Linux没有单独的重命名文件命令,而是通过\code{mv}命令来完成。
\end{attention}

进一步,如果前后两个参数都是目录会发生什么呢?很简单,\emph{就是将前面的目录移动至后面的目录},从效果上看,近似于将第一个参数的目录看作文件。

\begin{attention}
    与文件移动类似的操作,如重命名,对目录的移动同样成立。
\end{attention}

\subsection{如何删除文件\keyword{rm}}\label{subsec:文件操作-如何删除文件}

相比于移动文件需要两个参数,删除文件的命令\code{rm}只需要一个参数即可,也许你也能猜到这个参数的含义,即\code{rm <删除的文件路径>}。例如,要删除当前目录下的\code{INCAR}文件,只需要执行\code{rm INCAR}即可。同样的,你也可以使用更复杂的绝对路径或相对路径,例如,删除上一级目录下的\code{OUTCAR}文件,可以使用\code{rm ../OUTCAR}。

\begin{extend}
    与Windows不同,Linux删除文件通常是直接删除,而不是放在所谓的\emph{回收站}内。因此,在删除文件时务必小心。

    在有些版本的Linux(例如Ubuntu)当中,删除的文件被移动至\code{/home/<用户名>/.local/share/Trash/files}当中,这个目录起到的临时的\emph{回收站}功能,但你不应该寄希望于这个功能,而是仔细检查删除文件的正确性,并做好合适的备份。
\end{extend}

对于删除目录而言,情况有点特殊,需要使用\keywordin{rm}{rm -r}命令删除一个目录,此时后面所接参数为目录的路径,例如,删除当前目录下的\code{vasp}目录,则可以使用\code{rm -r vasp}。

\begin{attention}
    \code{-r}选项通常表示\emph{递归},例如,在\code{rm -r}当中,表示\emph{递归删除},从而达到删除一个目录的效果。在删除目录时会连同里面的所有内容都删除掉,因此需要特别小心。

    如果担心删除错误的文件,可以在选项中使用\code{-i}。\keywordin{rm}{rm -i}表示\emph{在删除时}询问是否删除。

    对于空目录而言,Linux还提供了一个命令\keyword{rmdir},其用法为\code{rmdir <目录路径名>},可以删除一个\emph{空目录}。
\end{attention}

\subsection{如何复制文件\keyword{cp}}\label{subsec:文件操作-如何复制文件}

复制文件的命令为\code{cp},其用法与移动文件\code{mv}几乎完全一样,无非就是将\emph{移动}改为\emph{复制}。简单来说,语法就是\code{cp <源文件路径> <目标文件路径>},类似于\ref{subsec:文件操作-移动文件}当中所介绍的重命名方法,使用\code{cp}命令同样可以做到复制的同时重命名。例如,\code{cp vasp/OUTCAR ../INCAR}表示将\code{vasp}目录下的\code{OUTCAR}文件复制到上一层目录,并重命名为\code{INCAR}

如果想要复制一个目录,也需要使用选项\keywordin{cp}{cp -r}。例如,\code{cp -r vasp/ python/}表示将\code{vasp}目录复制到当前目录并重命名为\code{python}。

\begin{attention}
    我们在上面的命令当中使用\code{vasp/}和\code{python/}表示两个目录。其中使用了符号\code{/}作为结尾,这个符号通常强调该路径是个目录。对于Linux本身而言,有没有这个符号并没有区别。例如,\code{cp -r vasp python}也可以表示上面的操作。我们这么写只是为了强调这两个路径是目录而不是文件。
\end{attention}

\subsection{一次性处理多个文件}\label{subsec:文件操作-一次性处理多个文件}

前面介绍的\code{rm},\code{cp},\code{mv},以及在\ref{sec:目录操作}一节所介绍的\code{mkdir},都是可以针对多个文件同时操作的。以\code{rm}为例,如果想同时删除多个文件,只需要在后面添加多个参数即可,其中参数之间以空格分割。例如,\code{rm INCAR KPOINTS}表示删除当前目录下的\code{INCAR}文件和\code{KPOINTS}文件。对于\code{mkdir}创建多个目录而言,也是一样的用法,例如,使用\code{mkdir vasp ML}表示在当前目录下创建\code{vasp}目录和\code{ML}目录。

对于\code{cp}和\code{mv}而言,情况稍有不同。它们自身就需要两个参数,第一个是源路径,第二个是目标路径。如果有多个文件需要处理,Linux默认\emph{最后一个路径为目标路径,前面的所有参数都是源路径}。例如,\code{cp INCAR KPOINTS POSCAR POTCAR ..}表示将\code{INCAR},\code{KPOINTS},\code{POSCAR}和\code{POTCAR}复制到上一级目录中。

\begin{attention}
    对于\code{cp}和\code{mv}而言,若一次性移动多个文件,则最后一个参数必须是目录。这就意味着不能进行重命名操作。
\end{attention}


\subsection{错误处理}\label{subsec:文件操作-错误处理}
% 请在本节列出可能遇见的错误与解决方法

\subsubsection{rmdir: failed to remove <路径名>: Directory not empty}

使用\code{rmdir}命令时,\emph{只能用来删除空目录}。当要删除的目录不是空目录时,执行该命令则会报错。使用\code{rm -r <路径名>}往往是删除非空目录的常见方法。

\subsubsection{cp: -r not specified; omitting directory <路径名>}

当使用\code{cp}复制目录时,需要添加\code{-r}选项。如果没有添加这一选项则会报错。

\subsubsection{cp: target <路径名> is not a directory}

这通常出现在尝试使用\code{cp}复制多个文件时,最后的参数\emph{必须是目录}。如果此时不是目录,则会报错。

\subsubsection{rm: cannot remove <路径名>: Is a directory}

类似于\code{cp}复制目录,使用\code{rm}删除目录时,也需要添加\code{-r}选项。特别地,对于一次性删除多个文件,如果在删除文件的同时也存在把目录删除的情况,也需要添加这一选项。

\subsubsection{rm: remove write-protected regular file <文件名>? }

当你尝试对没有权限(不可写)的文件进行删除时,会提示该错误。关于权限的内容,将在\ref{sec:文件权限管理}一节详细讨论。在Linux当中,是有方法对文件权限进行修改的,但这并不是一个明智的方法。仔细检查文件操作,遵守这些权限,不要“越界”,可以保证你“安全”地使用操作系统(不会引起系统崩溃等严重问题)。

如果你确实需要删除,则只需要输入\code{y}(表示“yes”)并回车即可;反之则输入\code{n}(表示“no”)。

