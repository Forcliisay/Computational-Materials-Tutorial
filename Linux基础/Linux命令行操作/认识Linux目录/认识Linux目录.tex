% 请在下方的大括号相应位置填写正确的节标题和标签,以及作者姓名
\section{认识Linux目录}\label{sec:认识Linux目录}
\sectionAuthor{Jiaqi Z.}

% 请在下方的item内填写本节知识点
\begin{Abstract}
    \item Linux命令格式
    \item 如何在Linux当中表示目录
    \item 绝对路径和相对路径
    \item 如何快速表示当前目录和上一级目录
\end{Abstract}

% 请在正文相应位置填写正确的小节标题(或小小节标题),同时将标签的“节标题”和“小节标题”改为实际内容

\subsection{命令格式}\label{subsec:认识Linux目录-命令格式}

与Windows使用可视化界面不同,Linux大多时候使用命令行(shell)进行操作。因此,在Linux的学习过程中,一个最重要的任务,就是掌握一些常见的Linux命令。对于大多数科研课题组而言,Linux系统都是在远程云端服务器上,因此在本地往往只需要一个终端程序即可连接到服务器。一些常见的终端软件包括Xshell、MobaXterm、甚至VS Code\footnote{对于VS Code而言,可能需要扩展插件(例如Remote-SSH)的支持}等。

\begin{attention}
    如果你熟悉其他操作系统,可能听闻过类似于Windows Server,或者Linux的Ubuntu这样的操作系统。明明也可以使用可视化界面,为什么在科研过程中从来不会用到它们呢?(更严谨地说,在远程服务器上)。实际上,当使用可视化界面进行远程连接时,所产生的网络资源消耗是巨大的,通常需要更大的带宽,而使用命令行就可以提高数据传输效率。此外,更重要的一点是,使用命令行可以很容易实现批量处理,这在后续的章节会介绍到。
\end{attention}

在Linux当中,输入命令通常采用的格式是\code{命令 [-选项] [参数]},其中中括号表示这个部分是\emph{可选的},即可以没有的。例如,当我们希望列出当前目录下所有文件时,可以使用\keyword{ls}直接输出,也可以使用\code{ls -l}以列表格式输出。

\begin{attention}
    在后面可能会看到选项有多个的情况,此时为了简化,可以将选项合并在一起。例如,\code{ls -l -a}可以简化为\code{ls -la}。

    命令与选项、参数之间是以空格进行分割,且这个空格不能省略。
\end{attention}

\subsection{目录表示方法}\label{subsec:认识Linux目录-目录表示方法}

在Linux当中,所有目录都是以根目录\code{/}为起点,任何目录都是根目录的子目录。根目录下存在一些固定的目录(这些目录通常有特定的含义),例如,在根目录下有一个叫做\code{bin}的目录(通常写作\code{/bin}),它存放的都是\emph{二进制文件},也就是系统可以执行的程序文件。

\begin{attention}
    在Linux当中,任何命令实际上都是可执行程序。你可以在\code{/bin}目录下看到后面所学的所有Linux终端命令。
\end{attention}

另一个比较重要的位置是家目录\code{/home},它存放的是用户个人文件。在这一目录下,你可以看到系统所注册的所有用户名。但是,这些文件夹大多数是无法查看的\footnote{这涉及到Linux操作权限的问题,通常来说,权限分为三组,即所有者权限、所属组权限和其他用户权限。对于\code{/home}目录下而言,所有目录都是对所有者(即这个用户本身)提供全部权限,而其他人无法访问、修改。}。对于用户自己的家目录,通常也可以表示为\code{\~}。通常来说,当你使用终端等连接登录时,默认的所在目录就是家目录\code{\~}

\subsection{绝对路径和相对路径}\label{subsec:认识Linux目录-绝对路径和相对路径}

任何目录在操作时都具有两种表示方式,\emph{绝对路径}和\emph{相对路径}。正如\ref{subsec:认识Linux目录-目录表示方法}所介绍的那样,任何目录都是从根目录开始的。因此在描述一个目录时,可以从根目录(即\code{/})开始。例如,若你在你的家目录下有一个叫做\code{vasp}的目录,那么它的绝对路径就是\code{/home/<你的用户名>/vasp}。

随着层级逐渐增多,这种表示方法也会越来越复杂,因此,在表示一个目录时,默认也可以从当前所在目录开始算起(即\emph{相对路径})。例如,若你刚刚进入终端,此时所在目录就是\code{\~}目录,即\code{/home/<你的用户名>}下,此时若想表示\code{vasp},则只需要使用\code{vasp}即可。

\begin{attention}
    在这种情况下,你可以将目录\code{vasp}理解为\code{<当前所在目录>/vasp},即等价于\code{/home/<你的用户名>/vasp}。

    千万不要写成\code{/vasp},它表示根目录下的\code{vasp}目录。如果你希望特别强调当前目录,可以使用符号\code{.}(一个点)表示“当前目录”,即可以写成\code{./vasp}
\end{attention}

然而,在这种情况下,回到\emph{当前目录的上一级目录}是麻烦的,即在目前所学范围内,只能使用绝对路径。好在Linux提供了一个命令:\code{..}(两个点)表示\emph{上一级目录}。因此,如果你当前处在目录\code{/home/<你的用户名>/vasp}当中,则\code{..}表示\code{/home/<你的用户名>}

同理,\code{../..}表示父目录的父目录,在上面的例子中即为\code{/home}目录。

\begin{attention}
    在终端当中,\code{..}(两个点)表示父目录(即上一级目录),而一个点\code{.}表示当前目录。

    这些符号(指令)在后续关于目录操作中都是可以使用的。
\end{attention}

看到这里,可以思考下面的问题:如果在你的家目录下有两个目录\code{python}和\code{vasp},此时你在\code{/home/<你的用户名>/vasp}目录下,如何可以快速表示\code{python}目录呢(不能使用绝对路径)?

\answer{\code{../python}即可表示\code{/home/<你的用户名>/python}目录}