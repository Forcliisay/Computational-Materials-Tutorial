% 请在下方的大括号相应位置填写正确的节标题和标签,以及作者姓名
\section{查看文件}\label{sec:查看文件}
\sectionAuthor{Jiaqi Z.}

% 请在下方的item内填写本节知识点
\begin{Abstract}
    \item Linux文件类型
    \item 如何查看Linux文件
\end{Abstract}

% 请在正文相应位置填写正确的小节标题(或小小节标题),同时将标签的“节标题”和“小节标题”改为实际内容

这一节看似知识点不多,但命令还是挺多的。因此,一节只讲这一部分内容完全足够了。

\subsection{Linux文件类型}\label{subsec:查看文件-Linux文件类型}

\begin{extend}
    在\ref{subsec:目录操作-显示目录文件}当中,我们介绍了\code{ls -l}命令可以以列表形式查看文件。当时仅仅提到,第一个字符如果是\code{d}则表示\emph{目录},如果是\code{-}则表示\emph{普通文件}。在这一部分,我们稍微详细介绍一下更多的文件类型。

    \begin{itemize}
        \item 普通文件(\code{-}):就是普通的文件,通常可以分为\emph{文本文件},\emph{可执行文件}和\emph{压缩文件}等;
        \item 目录(\code{d}):在Linux当中,目录也是一种文件,该文件下存放的是这一目录下的\emph{inode}号(又名\emph{索引节点})和文件名等信息。当执行打开文件时,Linux实际上是通过inode号找到当前文件所在block(8个磁盘扇区组成一个block),从而执行文件;
        \item 设备文件,又分为\emph{块设备文件}(\code{b})和\emph{字符设备文件}(\code{c})。其中前者可以以“块”为单位进行访问(例如硬盘,软盘等),而后者则是以“字节流”的方式访问(例如字符终端、键盘等)。一般来说,设备文件存放在\code{/dev/}目录下;
        \item 链接文件(\code{l}):一般情况下指的是符号链接(软链接),类似于Windows操作系统下的“快捷方式”。创建符号链接的方法是使用\keyword{ln}的\keywordin{ln}{ln -s}选项\footnote{相对地还有“硬链接”,直接使用\code{ln}即可。对于硬链接而言,二者本质上是一个文件(类似于做了备份),当其中一个删除时,另一个不会删除;当其中一个文件修改时,另一个也会同时修改},例如,\code{ln -s INCAR INCAR\_link}表示创建了一个指向\code{INCAR}文件的链接文件\code{INCAR\_link}。当源文件删除时,符号链接文件也会删除;
        \item 管道文件(\code{p}):通常用于进程间的通信,创建方法是\keyword{mkfifo}命令\footnote{也许你会疑惑为什么是fifo而不是管道的单词pipe。事实上,FIFO是一种数据缓存器执行方法,即“先进先出”(First In First Out)。作为数据缓存器,其与普通存储器的区别是没有外部读写地址线,这样使用起来非常简单,但缺点就是只能顺序写入数据,顺序的读出数据。数据地址在内部由指针自动加1实现,而不能通过地址线寻找地址。而Linux进程间的通信大多就是采用这种通信方式,这种方式也是管道的特性。相对的,还有一种LIFO,即“后进先出”(Last In First Out),通常“堆栈”(Stack)就是采用这种方法。},即\code{mkfifo fifo\_file}。
        \item 套接字文件(\code{s}):用于通信(尤其是网络上的通信)。简单来说,这是为了避免多个进程或多个TCP连接同时在一个TCP协议端口传输数据造成混淆。一般来说,套接字文件包含目的IP地址,传输层使用协议(TCP或UDP)和使用的端口号,利用套接字文件将三个参数组合起来,从而在传输过程中实现并发服务。
    \end{itemize}
\end{extend}

\subsection{查看文件内容\keyword{cat},\keyword{tac}}\label{subsec:查看文件-查看文件内容}

在整个Linux使用过程中,最关键的仍然是\emph{普通文件}和\emph{目录}。虽然其他文件对于操作系统本身而言也很重要,但对于非计算机相关专业的科研用户而言,其意义不大。上面的介绍仅仅是作为一个补充。下面地内容将专注在文件本身。首先一个关键的问题是:\emph{如何查看文件内容}。

\begin{attention}
    当然,从文件操作本身来说,第一件事应当是创建文件。但是,创建文件需要的内容较多(例如,需要一些\code{vi}编辑器的使用,可能还需要重定向命令,在后面的章节再详细介绍。
    
    如果是初学者,希望可以尽快上手的话,你可以试着在Windows本地用记事本创建一个文本文件,并在里面随意输入一些你喜欢的文字(建议使用英文,对于中文等非ASCII字符而言,可能会出现乱码。),然后利用远程终端将文件发送至服务器(对于MobaXterm而言,在终端左侧有一个目录列表,你可以直接将文件拖拽至相应的目录中;对于其他终端软件,请参考其软件具体的操作方法)。后面对文件的查看操作,都可以对这个文本文件进行。
\end{attention}

首先需要了解的是,如何查看完整的文件。在Linux当中,查看文件内容的命令是\code{cat},其基本用法是\code{cat <文件路径名>}例如,对于位于当前目录下的\code{INCAR}文件,可以使用\code{cat INCAR}查看其内容。

\code{cat}命令有一些常用选项,例如,可以使用\keywordin{cat}{cat -n}或\keywordin{cat}{cat -b}显示行号,二者的区别在于前者会显示所有行号,而后者只对有内容的行显示行号。如果文本中空行内容太多,可以使用\keywordin{cat}{cat -s}对空行进行压缩,使其缩减为一个空行。

相对地,命令\code{tac}也是查看所有内容,只不过它是\code{从最后一行倒着输出}。可以看出,\code{tac}本身就是命令\code{cat}倒着写。例如,\code{tac INCAR}表示从最后一行开始输出\code{INCAR}文件。

\begin{attention}
    命令\code{cat}并不是单词“猫”的意思,而是连接concatenate的缩写。正如单词所表示的那样,\code{cat}最原始的功能,是连接多个文件。例如,有一个文件叫\code{a},另一个文件叫\code{b},执行命令\code{cat a b},则会将两个文件内容按照顺序连接起来并输出。
\end{attention}

\subsection{关于文件后缀名}\label{subsec:查看文件-关于文件后缀名}

对于熟悉Windows的用户而言,看到上面(包括之前的所有示例)也许都会有一个疑惑:在Linux当中,文件名为什么没有后缀?事实上,后缀名的重要性仅仅是Windows操作系统给你的一个“错觉”,让你误以为\emph{后缀名很重要}。事实上,\emph{Windows操作系统的文件名后缀并不会影响这个文件本身}。

例如,在Windows操作系统下创建一个文本文件,后缀名为\code{.txt},此时试着将其后缀名改为\code{*.mp3}或者\code{*.jpg}等,并再次在打开方式中用记事本打开(如果使用默认的打开方式一定会出现错误,例如音乐播放器或者图片查看器等)。可以发现,用记事本打开的结果与之前的文本文件内容是完全一样的。

\begin{attention}
    虽然表示后缀名的\code{.}可以任意放置,但有一个地方比较特殊--文件名开头。对于以\code{.}开头的文件名而言,它表示的含义是隐藏文件(这在\ref{subsec:目录操作-关于隐藏文件}一节介绍过了)
\end{attention}

对于Windows操作系统而言,使用后缀名往往是为了决定文件的打开方式(取决于Windows特有的\emph{注册表});而Linux文件大多都是文本文件(甚至系统配置也是文本文件),因此在Linux当中,文件后缀名就变得不重要了。也正因如此,在Linux当中你可以类似于Windows后缀名的方式创建任何的后缀(\code{*.jpg},\code{*.xyz}甚至\code{*.zjq},\code{*.ykn}都是可以的),在Linux看来,它们仅仅是文件名的一部分。

甚至,在Linux当中,大多时候文件都是没有后缀名的。这也就是之前的\code{INCAR}和\code{OUTCAR}为什么没有后缀。对于从Windows创建的文本文件上传至服务器而言,可能还留有所谓的后缀名\code{*.txt},你完全可以使用\code{mv}命令将后缀名删去,丝毫不影响文件本身和其他命令的运行。

\subsection{按页查看文件\keyword{more},\keyword{less}}\label{subsec:查看文件-按页查看文件}

使用\code{cat}和\code{tac}查看文件,都是“一股脑”输出到终端里,对于比较短的文件而言,这种方法是可行的;如果这个文件很长,则要上下翻页就会比较繁杂。

对于多页的文件而言,Linux可以使用\code{more}命令查看。基本用法是\code{more <文件路径名>}。例如,使用\code{more ../band/OUTCAR}就可以查看上一级目录下的\code{band}目录下的\code{OUTCAR}文件。在查看过程中,可以\emph{使用空格进行翻页,使用回车进行下一行查看}。

在查看过程中,可以随时使用\code{q}键退出。

对于一些需要往回翻页查看的文件,可以使用\code{less}命令。基本调用格式与\code{more}类似,即\code{less <文件路径名>}。与\code{more}不同的是,\code{less}命令可以向上翻页(使用\code{Page Up}键或者\code{b}键)\footnote{除次之外,也可以使用\code{d}向后翻半页,使用\code{u}向前翻半页。}

\begin{attention}
    对于\code{more}而言,实际上也可以通过\code{b}键实现向前翻一页的效果。但相比于\code{less}而言,\code{more}的自由性并不是太高。而且,使用\code{b}向前翻页的效果对于管道文件无法实现。

    此外,\code{less}还有更复杂的“搜索功能”,例如,可以使用符号\code{/<字符串>}的方法实现向下搜索,使用符号\code{?<字符串>}的方法实现向上搜索。同时,\code{less}的其他命令都是在显示文件后的操作,并不是类似于之前的“选项”(即使用\code{-}的形式),这种方法与vi的使用类似。
\end{attention}

无论是\code{more}还是\code{less},都可以使用\code{q}键退出显示文件。

\subsection{取头部\keyword{head}和取尾部\keyword{tail}}\label{subsec:查看文件-取头部和取尾部}

有时,可能会希望仅仅查看一个文件的开头或者结尾。此时可以使用Linux操作系统下的\code{head}和\code{tail}命令。这两个命令的基本调用方法都是一样的,即\code{head <文件路径名>}和\code{tail <文件路径名>}。例如,使用\code{head POSCAR}就可以查看当前目录下\code{POSCAR}文件开头几行,同理,使用\code{tail relax/OSZICAR}就可以查看\code{relax}目录下的\code{OSZICAR}文件结尾几行。

\begin{attention}
    通常情况下,直接调用\code{head}和\code{tail}得到的都是开头(或结尾)10行的内容。在有些时候,可能会希望输出更多行,或者少输出几行避免混乱。此时可以使用参数\keywordin{head}{head -n}和\keywordin{tail}{tail -n}实现,其基本格式为\code{head -n <行数> <文件路径名>}和\code{tail -n <行数> <文件路径名>},这一选项表示输出指定的行数。例如,\code{head -n 5 POSCAR}可以查看\code{POSCAR}文件开头5行。对于\code{tail}同理。

    除此之外,\code{head}和\code{tail}还提供了选项\keywordin{head}{head -c}和\keywordin{tail}{tail -c},表示输出开头(或结尾)多少个字符的内容,格式与上面\code{-n}选项类似,即\code{head -c <字符数> <文件路径名>}和\code{tail -c <字符数> <文件路径名>}。
\end{attention}

\subsection{错误处理}\label{subsec:查看文件-错误处理}
% 请在本节列出可能遇见的错误与解决方法

\subsubsection{cat: <文件名>: Is a directory}

\code{cat}命令仅限于查看文件内容,若后面所接内容为一个目录,例如,\code{cat vasp/}则会报错

\subsubsection{输入\code{cat}命令后忘记输入文件名直接回车,输入文件名后结果只输出了文件名,并没有输出内容}

当直接调用\code{cat}而没有接任何参数时,表示将终端标准输入所读取到的内容输出到终端。对于普通调用\code{cat <文件路径名>}而言,是将读取到的文件输出到终端。若没有任何参数,则会读取后面输入的内容。

退出的方法则是使用\code{ctrl+d}键结束当前输入,或者使用\code{ctrl+c}键强制终止当前命令。

\subsubsection{cat: <文件名>: No such file or directory}

文件路径不存在,检查一下路径(尤其是当前工作路径)是否正确。

\subsubsection{head(或tail): invalid number of lines: <文件名>}

当你使用\code{head -n}或\code{tail -n}时,后面的行数是必须提供的一个参数。若没有提供行数,则会报错。

\subsubsection{head(或tail): cannot open <文件名> for reading: No such file or directory}

文件路径不存在,检查一下路径(尤其是当前工作路径)是否正确。

\subsubsection{head(或tail): error reading <文件名>: Is a directoryy}

类似于使用\code{cat}打开目录,使用\code{head}或\code{tail}打开目录就会报这种错误。

\subsubsection{使用\code{more}查看文件,输出*** <文件名>: directory ***}

这是因为试着用\code{more}查看目录而不是文件。

\subsubsection{使用\code{less}查看文件,输出许多奇怪的路径,不是想要的内容}

如果你仔细看一下里面的内容就会发现,当你用\code{less}查看目录时,输出的是这个目录下所有的文件和目录(包括隐藏文件)。事实上,使用\code{less <目录路径>}得到的结果和使用\code{ls -l <目录路径>}是一样的。只不过前者是单独输出的,而后者是直接输出在终端里。

\subsubsection{Missing filename ("less --help" for help)}

在调用\code{less}时忘记提供文件路径了。

\subsubsection{more: bad usage Try 'more --help' for more information.}

与上面的错误类似,在调用\code{more}时忘记提供文件路径了。