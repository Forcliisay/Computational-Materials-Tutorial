% 请在下方的大括号相应位置填写正确的节标题和标签,以及作者姓名
\section{目录操作}\label{sec:目录操作}
\sectionAuthor{Jiaqi Z.}

% 请在下方的item内填写本节知识点
\begin{Abstract}
    \item 如何显示当前目录下所有文件
    \item 如何创建目录
    \item 如何切换至其他目录
\end{Abstract}

% 请在正文相应位置填写正确的小节标题(或小小节标题),同时将标签的“节标题”和“小节标题”改为实际内容

\subsection{显示目录文件\keyword{ls}}\label{subsec:目录操作-显示目录文件}

在这一节以及下一节,我们将讨论如何对目录和文件做基本的操作。无论是哪一种,一个最基本的前提是\emph{知道当前目录有哪些文件和目录},从而才能进行后续操作(例如编辑、删除、移动、进入目录等)

在Linux当中,列出一个目录下所有文件使用的是\code{ls}命令。在没有任何参数与选项的前提下,它输出的结果就是\emph{当前所在目录下的所有文件和目录}。以\ref{subsec:认识Linux目录-绝对路径和相对路径}一节最后的例子为例,家目录下有\code{vasp}和\code{python}两个目录,当在家目录下执行\code{ls}命令时,结果如下:

\begin{lstlisting}[language=bash]
$ ls
vasp  python
\end{lstlisting}

同时,\code{ls}支持在后面添加一个参数表示要输出的目录。例如,在这一例子下,若在家目录当中执行命令\code{ls vasp},将会输出\code{vasp}目录下的所有文件和目录。利用\code{..}表示上一级目录的用法,若当前处在\code{\texttilde/vasp}目录下,使用\code{ls ..}便可得到上一级目录(即家目录)下的所有文件和目录。

\subsubsection{\keywordin{ls}{ls -l}}

下面介绍两个常见的\code{ls}选项,首先是\code{-l}选项,它表示\emph{以列表形式输出结果}。例如,还是上面的例子,使用这一命令的结果为:

\begin{lstlisting}[language=bash]
$ ls -l
total 0
drwxrwxr-x 2 zjq zjq 6 Aug 12 16:35 python
drwxrwxr-x 2 zjq zjq 6 Aug 12 16:35 vasp
\end{lstlisting}

\begin{extend}
每一个文件的输出结果可以分为9个部分,分别是:权限、文件硬链接数或目录子目录数、拥有者用户名、拥有者所在组、文件大小、文件修改月份、日期、时间、文件名。

关于权限,可以将其分成四部分:第一部分(一个字符)表示文件类型(这里的d表示目录),第二部分(三个字符)表示拥有者权限(rwx表示可读可写可执行),第三部分(三个字符)表示组用户权限,第四部分(三个字符)表示其他用户权限(r-x)表示可读,可执行但不可编辑。

对于文件硬链接数和目标子目录数,对于初始创建的文件而言,通常为1,而对于目录而言,默认为2(因为有两个子目录\code{.}和\code{..})
\end{extend}

有时,也可以使用\keyword{ll}代替指令\code{ls -l},其二者是完全等价的。

\subsubsection{\keywordin{ls}{ls -a}}

\code{-a}选项表示\emph{列出所有文件,包括隐藏文件}。例如,在\code{\texttilde/vasp}目录下,使用\code{ls -a}命令,结果为:

\begin{lstlisting}[language=bash]
$ ls -a
.  ..extend
\end{lstlisting}

\begin{extend}
    正如前面所介绍的那样,任何一个空目录都会默认有两个隐藏目录--自身和它的上一级目录。而这也解释了\ref{subsec:认识Linux目录-绝对路径和相对路径}一节所介绍的\code{.}和\code{..}的本质,它们实际上就是任何当前目录下的两个子目录。
\end{extend}

\begin{attention}
    前面所介绍的\code{-l}选项和\code{-a}选项是可以合并使用的,此时可以将两个选项之间以空格分割,如\code{ls -l -a},或者将两个选项写在一起\code{ls -la}

    当选项写在一起时,选项的排列顺序不重要。

    与最开始介绍\code{ls}后面加参数表示目录一样,带有选项的\code{ls}同样可以在后面添加参数,例如,\code{ls -a vasp}表示列出当前目录下的\code{vasp}子目录下的所有文件和目录(包括隐藏文件)
\end{attention}

\subsection{关于隐藏文件}\label{subsec:目录操作-关于隐藏文件}

\begin{extend}
    隐藏文件是指在文件名前面加上\code{.}的,例如\code{.bashrc}。

    隐藏文件在Linux当中的常见用途有:
    \begin{itemize}
        \item 配置文件
        \item 临时文件
        \item 缓存文件
        \item 等
    \end{itemize}

    总而言之,隐藏文件是为了防止误操作而存在的。(这可能与一些人认为的“隐藏文件是避免别人看到”不同)事实上,哪怕在Windows操作系统中,隐藏文件也是存在且方便查看的\footnote{在Windows操作系统中,可以通过右键-属性-隐藏的方式将文件或文件夹设置为隐藏;相对地,对Windows10操作系统而言,可以通过文件夹菜单栏的“查看”-“隐藏的项目”找到那些隐藏文件。只不过在Linux当中,隐藏文件使用前面加点\code{.}的方式设置,但无论如何,隐藏文件永远不是不让别人看见的方法,如果想达成这一目的,正确的方法是设置权限。}
\end{extend}

\subsection{创建目录\keyword{mkdir}}\label{subsec:目录操作-创建目录}

如果所有操作都在家目录下进行,那文件管理就太复杂了。试想一下,在科研里面算了好几年的结果,全部“一股脑”堆在一起,既难找,也容易忘记当时是做了什么。因此,一个好的目录管理至关重要。而前提,就是知道如何创建目录。

在Linux当中,创建目录的方法是使用\code{mkdir}命令。与前面介绍的\code{ls},以及后面要介绍的\code{cd}不同的是,\code{mkdir}必须带有一个参数,表示\emph{创建的目录路径}。对于刚开始接触的初学者,一个最简单的命令格式是:\code{mkdir <目录名>},其中表示在当前目录下创建一个名为\code{<目录名>}的目录。例如,希望在当前目录下创建一个名为\code{ML}的目录,则可以使用命令\code{mkdir ML}。

正如前面所介绍的路径,\code{mkdir}后面的路径也可以是绝对路径或相对路径。无论是哪种形式,其含义是一样的,即\emph{在你所描述的路径下创建目录}。利用这种方法,你可以在更远的层级关系下创建目录。例如,在\code{\texttilde/vasp/lattice/Fe}目录下创建\code{\texttilde/python/ML/plot}目录。

\begin{attention}
    你所写的路径名,应当是你所要创建的目录。这句话似乎有点绕,举个例子,如果你希望在\code{/home/zjq/vasp}下创建一个名为\code{lattice}的目录,则你需要运行的命令是\code{mkdir /home/zjq/vasp/lattice}。注意到,后面的路径实际上就是你要创建的目录。
\end{attention}

\subsection{切换目录\keyword{cd}}\label{subsec:目录操作-切换目录}

在Linux当中,切换目录使用的命令是\code{cd},通常来说,后面需要配合一个参数,表示\emph{要切换到哪里}。例如,使用命令\code{cd /home}则是将当前目录切换到\code{/home}目录下。配合以\code{..},可以使用\code{cd ..}切换到上一级目录。

思考:如果使用\code{cd .},会得到什么结果?

\answer{这个命令的含义是\emph{切换到当前目录},最终效果就是什么也不发生}

特殊的,对于家目录而言,除了可以使用\code{cd \texttilde}外,Linux也支持直接使用\code{cd},不添加任何参数实现这一功能,即二者是等价的。

\subsection{错误处理}\label{subsec:目录操作-错误处理}
% 请在本节列出可能遇见的错误与解决方法

\subsubsection{-bash: cd: <目录名>: Not a directory}

\code{cd}后面的参数必须是目录,不能是文件。如果参数是文件,则会报该错误。

如果不知道哪个是目录,哪个是文件,可以使用\code{ls -l}查看第一个字符(文件类型),如果第一个字符是\code{d},则表示目录,如果是\code{-},则表示文件\footnote{在一些比较新的终端程序中,可能会将文件和目录以不同颜色区分。例如,在MobaXterm当中,默认情况下文件是白色,目录是蓝色。当然,这些颜色设置都是可以通过Settings-Terminal-Default color settings设置颜色主题,这里所说的这一例子为“Dark background / Light text”主题}。例如,

\begin{lstlisting}[language=bash]
$ ls -l
total 4
-rw-rw-r-- 1 zjq zjq 4 Aug 12 17:12 INCAR
drwxrwxr-x 2 zjq zjq 6 Aug 12 16:35 python
drwxrwxr-x 2 zjq zjq 6 Aug 12 16:35 vasp
\end{lstlisting}

表示\code{INCAR}是文件,而\code{vasp}和\code{python}是目录。如果执行了\code{cd INCAR},则会报错。

\subsubsection{-bash: cd: <目录名>: Permission denied}

这表明你尝试进入一个你没有权限的目录。例如,在\code{/home}目录下,有\code{ljk}和\code{zjq}两个目录,分别表示两个用户。如果执行\code{ls -l}则会发现:

\begin{lstlisting}[language=bash]
$ ls -l
total 32
drwx------  13 ljk    ljk    4096 Aug  5 17:34 ljk
drwx------  75 zjq    zjq    4096 Aug 12 17:12 zjq
\end{lstlisting}

很显然,每个目录只有目录拥有者自己可以访问。例如,作为用户\code{zjq},当尝试执行\code{cd ljk}时,则会报错。

\begin{extend}
    这种情况有一个特例:root用户。对于root用户而言,可以进入任何目录。但通常来说,root用户是由服务器管理者所持有的,作为一般用户而言,不需要也不应该进入没有权限的目录,或者执行没有权限的操作\footnote{“删库跑路”这件事对于一般用户来说是不可能的}。
\end{extend}

\subsubsection{mkdir: cannot create directory <目录名>: No such file or directory}

虽然我们说可以用绝对路径或相对路径在更远的层级关系下创建目录。但这一操作的前提是,\emph{这个目录的上一级目录需要存在}。例如,当你执行\code{mkdir vasp/lattice/Fe}时,首先需要确保目录\code{vasp}和\code{vasp/lattice}存在,才会创建\code{vasp/lattice/Fe}。如果你要创建的目录其上一级目录不存在,则会报错。

一个很自然的解决方法是:\emph{一层一层创建}。这种方法虽然麻烦,但可以确保目录是清晰的\footnote{如果你确实想要一个快捷的方法,可以使用选项\code{-p}。这一选项可以在遇到没有的目录时自动为你创建。例如,上面的例子也可以直接使用\code{mkdir -p vasp/lattice/Fe},但这一操作需要保证输入内容是正确的。一旦有内容输错,则极有可能造成目录结构混乱。}。

\subsubsection{mkdir: missing operand}

很显然,你在使用\code{mkdir}时没有给任何参数。正如\ref{subsec:目录操作-创建目录}所说的那样,在调用\code{mkdir}时\emph{必须提供一个参数表示要创建的目录路径}。