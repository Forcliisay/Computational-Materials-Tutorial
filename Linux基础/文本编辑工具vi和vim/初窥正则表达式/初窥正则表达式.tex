\section{初窥正则表达式}\label{sec:初窥正则表达式}
\sectionAuthor{Jiaqi Z.}

\begin{Abstract}
    \item 什么是正则表达式
    \item 如何使用简单的正则表达式进行查找和替换
\end{Abstract}

\subsection{关于正则表达式}\label{subsec:初窥正则表达式-关于正则表达式}

在\ref{sec:查找与替换}一节当中,我们提到过,vi的查找和替换相比于其他文本编辑器都稍显复杂。而这一节所介绍的\emph{正则表达式},则是其十分强大的功能之一。

简单来说,正则表达式是\emph{一种用于匹配和操作文本的强大工具,它是由一系列字符和特殊字符组成的模式,用于描述要匹配的文本模式}。借助于正则表达式,我们可以很方便对许多具有相同模式的字符串进行匹配与处理。例如,对于\code{ENCUT=200}和\code{ENCUT=400},从字符串本身来看是不同的,但二者具有相同的模式(\code{ENCUT=}加上一系列整数字符)。因此,可以使用正则表达式进行批量处理。

在Linux当中,正则表达式是相对比较复杂的内容。在这一节只是简单介绍一下基本用法,对于更完整的内容,将在后面章节进行介绍。

\subsection{元字符}\label{subsec:正则表达式-元字符}

正则表达式最有特色的部分,就是可以使用\emph{元字符}来匹配一系列特定的字符。在介绍一些复杂的元字符之前,先熟悉一个最简单的符号,\code{[]},在中括号里面,可以放入一些字符。正则表达式将会\emph{匹配这些字符当中的一个}。例如,对于字符串“hello”,使用正则表达式\code{[aeiou]}就可以匹配到字符串里面的所有元音字母。

\begin{attention}
    在vi当中,可以使用正常的查找方式和替换方式,只不过需要在输入查找的内容时使用正则表达式。简单说,你可以将正则表达式看作是一个表达多个字符串集合的方式,而可以使用这种方式一次性对这个集合内的每一个元素进行查找和替换。这样的话,其使用方法就与普通的查找和替换基本无异了。

    同时,特别需要注意的一点是,在vi当中,有一些符号(后面会提到)与Linux本身的正则表达式不同(Linux的命令行本身也是支持正则表达式的),通常区别在于是否添加一个反斜杠(\code{$\backslash$})。后面遇到时会特别指出。
\end{attention}

在上面的例子中,我们可以直接在vi当中直接使用\code{/[aeiou]}实现对所有元音字母的查找。

在使用\code{[]}时,可以使用\code{-}对特定范围内的字符进行查找。例如,使用\code{[a-h]}表示对a到h之间的所有字母(小写字母)进行查找。常用的还有,使用\code{[A-Z]}表示对所有大写字母进行匹配,\code{[a-z]}表示对小写字母进行匹配,\code{[0-9]}表示对所有阿拉伯数字进行匹配。

\begin{extend}
    也许你会有疑问:这个范围是按照什么排序的?在计算机当中,这些字符都是根据ASCII码将其转化为二进制存储在计算机内。因此,这里的排序也是根据每一个字符所对应的ASCII码排序的。

    ASCII(American Standard Code for Information Interchange,美国信息交换标准代码)是基于拉丁字母的一套电脑编码系统。它主要用于显示现代英语,而其扩展版本延伸美国标准信息交换码则可以部分支持其他西欧语言,并等同于国际标准ISO/IEC 646。

    ASCII 由电报码发展而来。第一版标准发布于1963年 ,1967年经历了一次主要修订,最后一次更新则是在1986年,至今为止共定义了128个字符;其中33个字符无法显示(一些终端提供了扩展,使得这些字符可显示为诸如笑脸、扑克牌花式等8-bit符号),且这33个字符多数都已是陈废的控制字符。控制字符的用途主要是用来操控已经处理过的文字。在33个字符之外的是95个可显示的字符。

    例如,0的ASCII码为48,A的ASCII码为65,而a的ASCII码为97。因此,可以使用\code{[0-a]}匹配到大写字母\code{A}。
\end{extend}

同时,中括号里面的字符是可以组合使用的,例如,可以使用\code{[A-Za-z]}表示所有的字母。那如果希望表达所有的字母和数字呢?

\answer{\code{[A-Za-z0-9]}}

除此之外,对于一些常见的字符,为其设置了特殊的符号,例如,\code{$\backslash$d}就表示\emph{所有的数字字符},\code{$\backslash$w}表示\emph{所有的字母、数字和下划线},也就等价于\code{[A-Za-z0-9\_]}。

而在使用中括号时,也可以使用符号\code{\^}进行\emph{反选}。例如,使用\code{\^[A-Z]}表示排除所有大写字母的字符。

\subsection{总结}\label{subsec:初窥正则表达式-总结}

本节简单介绍了一些常见的元字符,并可以将其用于查找和替换。例如,在本节开头所介绍的\code{ENCUT=200}和\code{ENCUT=400},使用正则表达式可以直接表示为:\code{ENCUT=$\backslash$d$\backslash$d$\backslash$d}\footnote{事实上,它还有更简洁的表示方法\code{ENCUT=$\backslash$d$\backslash$+},但碍于本节的内容,详细的含义将放在后面章节介绍。}。

正如最开始所说的那样,正则表达式的功能远不止此,对于更复杂的部分(例如,目前使用\code{[]}只能匹配一个字符,如何匹配多个字符?),将在后面的章节进行更加详细的介绍。

\subsection{错误处理}\label{subsec:初窥正则表达式-错误处理}

\subsubsection{如何查找如\code{[hello]}这样的字符串?}

在正则表达式当中,已经将中括号作为特殊符号使用。因此,如果想查找带有中括号的字符串,则需要将中括号前面添加一个反斜杠\code{$\backslash$}表示中括号这一字符本身。例如,对于上面的例子,如果直接使用\code{[hello]}表示匹配这5个字母(实际为4个)当中的任意一个字符;而使用\code{$\backslash$[hello$\backslash$]}或者\code{$\backslash$[hello]}都可以表示字符串“[hello]”