\section{查找与替换}\label{sec:查找与替换}
\sectionAuthor{Jiaqi Z.}

\begin{Abstract}
    \item 使用\code{vi}的\code{/}和\code{?}进行字符串查找
    \item 使用\code{vi}进行字符串的替换
\end{Abstract}

对于一个现代文本编辑器,一个最基本的功能就是对某一特定的字符串进行查找,以及将其替换为另一字符串。相比于其他在Windows操作系统中常见的文本编辑器(无论是记事本、word、还是VS Code等),Linux的vi编辑器下的查找和替换都显得更加复杂。这确实可能带来了一些学习上的困难,但随着使用场景逐渐复杂,你会发现这种代码式的操作的便利性。

\subsection{查找}\label{subsec:查找与替换-查找}

首先先来了解如何对一个字符串进行查找。在vi当中,查找的方法是使用\keywordin{vi}{/}或者\keywordin{vi}{?},其基本格式为\code{/[要查找的字符串]}或者\code{?[要查找的字符串]}。例如,在当前文件中查找\code{Hello},可以输入\code{/Hello},然后回车。

\begin{attention}
    在输入字符串时,vi会同时在文本内将所有匹配的字符串进行高亮(即便没有按回车)。

    \code{/}和\code{?}的作用都是查找字符串,二者的区别在于,\code{/}是从当前光标开始向后查找,而\code{?}是向前查找。当输入完成后,点击回车,光标会自动定位到最近的相应位置。若要切换,则可以使用\keywordin{vi}{n}查找下一个或者使用\keywordin{vi}{N}查找上一个。
\end{attention}

\subsection{替换}\label{subsec:查找与替换-替换}

相比于查找命令,vi中的替换命令就显得更加复杂了。最基本的命令是\keywordin{vi}{s},但通常会配以更多的命令(类似于参数)。一般来说,替换命令可以用下面的方式表示:\code{:<开始行号>,<结束行号>s[分隔符][要替换的字符串][分隔符][替换为的字符串][分隔符]<g>}。其中\code{<开始行号>}和\code{<结束行号>}都是可选的,若省略则表示\emph{只对当前行进行替换}。命令结尾的\code{<g>}也是可选的,表示对所有进行替换,若省略则只替换第一个(每一行或当前行,取决于是否有行号)。

同时,在替换时需要使用\code{[分隔符]}对字符串进行分割,通常情况下习惯于使用\code{/}表示,但在一些特殊的情况下(例如要替换的字符串内带有这一字符),则可能会将其改为其他分隔符。命令当中所有出现分隔符的地方都需要\emph{统一}。

下面是一些例子,例如,若希望将当前行的\emph{第一个}“hello”替换为“bye”,则需要命令\code{:s/hello/bye/},若希望对所有字符串进行替换,则使用\code{:s/hello/bye/g}。

若希望对第一行到第三行的所有“hello”进行替换,则使用\code{:1,3s/hello/bye/g},若没有最后的\code{g},则表示仅对\emph{第一行到第三行每一行里面的第一个字符串进行替换}。

如果希望对第一行到最后一行的所有“hello”进行替换,则使用\code{:1,\$s/hello/bye/g}。其中,\keywordin{vi}{\$}表示\emph{最后一行}。

\begin{attention}
    在vi当中,数字可以具有\emph{重复若干次}的含义。例如,在前面所介绍的\code{x}表示删除当前光标所在字符,若前面加上一个数字,则表示重复这一操作多少次(即删除多少字符),例如,\code{10x}表示删除10个字符。

    同时,在vi当中,往往使用\code{\$}表示\emph{最后}的意思。例如,在普通模式下直接输入\code{\$}则直接跳转到\emph{这一行最后一个字符},类似的,输入\code{0}则跳转到这一行第一个字符。输入\code{:\$}可以直接跳转到文件最后一行。
\end{attention}


\subsection{错误处理}\label{subsec:查找与替换-错误处理}
% 请在本节列出可能遇见的错误与解决方法

\subsubsection{查找(替换)完之后字符串总是高亮显示,怎么将其关闭}

使用\code{:noh}命令。

\subsubsection{E488: Trailing characters}

这可能是在输入命令时使用了错误的格式。请仔细检查使用的命令(尤其是替换命令)的格式

\subsubsection{想要替换,却发现把光标上的字符删除了}

这是因为在使用替换命令时,前面需要有冒号\code{:}。若没有添加这一符号,直接使用\keywordin{vi}{s}则意味着\emph{删除当前字符并插入}