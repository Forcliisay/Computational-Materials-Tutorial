% 请在下方的大括号相应位置填写正确的节标题和标签,以及作者姓名
\section{VASP计算能带过程}\label{sec:VASP计算能带过程}
\sectionAuthor{Jiaqi Z.}

% 请在下方的item内填写本节知识点
\begin{Abstract}
    \item 如何使用VASP计算PBE能带
\end{Abstract}

在本节,我们将详细讨论如何使用VASP计算能带。我们先讨论最简单的PBE能带计算过程,旨在通过这一流程,掌握计算能带的完整步骤。在这一基础上,后面将会详细讨论精度更高的计算方法(如HSE能带计算等)。

\begin{attention}
    使用PBE计算能带往往会得到较小的带隙,如果你使用数据库或文献中的能带图进行复现,可能会得到与文献不同的带隙。这一点是PBE泛函计算能带所固有的缺陷,
\end{attention}

\subsection{结构优化}\label{subsec:VASP计算能带过程-结构优化}

在这一部分计算能带时我们使用\ch{SiO2}为例进行分析。所使用数据库来源自Materials Project\footnote{https://legacy.materialsproject.org/materials/mp-546794/}。

\ch{SiO2}的结构文件\code{POSCAR}如下所示:

\begin{lstlisting}[caption=POSCAR]
Si2 O4
1.0
        5.1358423233         0.0000000000         0.0000000000
        0.1578526541         5.1334159104         0.0000000000
        -2.6468476750        -2.5667081359         3.5753437737
    Si    O
    2    4
Direct
        0.750000000         0.250000000         0.500000000
        0.000000000         0.000000000         0.000000000
        0.787033975         0.625000000         0.662033975
        0.875000000         0.212965995         0.837966025
        0.962966025         0.125000000         0.337965995
        0.375000000         0.037034001         0.162034005
\end{lstlisting}

\begin{attention}
    在计算能带时,大多数时候我们只讨论原胞的计算。因此,在使用Materials Project等数据库导出结构时,应当优先导出原胞结构(Primitive Cell)。

    对于无法导出原胞的情况,可以借助于其他程序或脚本文件。以vaspkit为例,借助于\code{vaspkit-602},可以得到\code{PRIMCELL.vasp}文件,将其重命名为\code{POSCAR}文件即可。
\end{attention}

\begin{extend}
    对于一些特殊情形(例如需要掺杂等情况),不得不使用超胞进行计算。如果确实需要计算能带结构,往往需要对能带进行\emph{反折叠}以得到更清楚的图像。我们将在后面的部分对这一技术进行讨论。

    目前,我们所讨论的结构都是原胞。
\end{extend}

在计算能带之前,首先需要对材料进行结构优化。为得到结构优化所用\code{INCAR}文件,使用\code{vaspkit-101-LR}生成。同时调整其中的部分参数,修改后的\code{INCAR}文件如下:

\begin{lstlisting}[caption=INCAR]
Global Parameters
ISTART =  1
LREAL  = .FALSE.
ENCUT  =  600
PREC   =  Accurate
LWAVE  = .TRUE.
LCHARG = .TRUE.
ADDGRID= .TRUE.
    
Lattice Relaxation
NSW    =  300
ISMEAR =  0
SIGMA  =  0.05
IBRION =  2
ISIF   =  3
EDIFFG = -1.5E-02
\end{lstlisting}

其中,需要特别注意并调整的是:

\begin{itemize}
    \item \keyword{ENCUT}:截断能。通常设置为600或更高,但更高的截断能往往意味着更长的机时。同时,在后续所有计算中,截断能应当保持不变。
    \item \keyword{ISIF}:表示优化方式。对于一般的结构优化,通常设置为\code{ISIF=3}表示\emph{优化原子坐标和晶格参数}。对于一些特殊的材料(如二维材料),一些晶格参数可能不希望发生变化,此时可以设置\keyword{OPTCELL}文件,其内容为$3\times3$的矩阵,分别对应\code{POSCAR}当中的晶格参数坐标。其元素可以是0(表示不优化该坐标)或1(表示优化该坐标)。对于晶格参数不变的情况,可以设置为\code{ISIF=2}表示\emph{只优化原子坐标}。
    \item \keyword{EDIFFG}:表示优化收敛标准。其中正数表示\emph{能量收敛标准}(即能量变化小于这一数值时停止计算),而负数表示\emph{力收敛标准}(原子作用力小于这一数值的绝对值时停止计算)。
    \item \keyword{NSW}:表示\emph{最大离子步}。当优化离子步达到设定数值时停止计算(此时往往未达到收敛标准,需要重新计算)。或者,当\code{EDIFFG=0}时,计算达到设定\code{NSW}时停止计算。
\end{itemize}

对于\code{KPOINTS}文件,在结构优化时可以使用“自洽计算”的K点,使用\code{vaspkit-102}生成,通常设定Gamma点(2),选择密度时通常设定为0.02-0.04即可。

本例使用\code{vaspkit-102-2-0.02}生成\code{KPOINTS}文件如下所示:

\begin{lstlisting}[caption=KPOINTS]
K-Spacing Value to Generate K-Mesh: 0.020
0
Gamma
  12  12  14
0.0  0.0  0.0
\end{lstlisting}

\begin{attention}
    使用vaspkit生成K点的同时,脚本会同步生成赝势文件\code{POTCAR}。但为了确保生成文件的正确性,建议使用\code{grep TITEL POTCAR}查看赝势文件是否正确(与\code{POSCAR}文件相比较)\footnote{我也不知道为什么它是“TITEL”而不是“TITLE”,如果实在记不住的话用\code{TIT}也能搜索到对应内容。}。
\end{attention}

将以上文件放置在一个目录下,提交任务计算后得到\code{CONTCAR}文件,即为优化后得到的结构文件。

\subsection{自洽计算}\label{subsec:VASP计算能带过程-自洽计算}

计算完成后,新建一个目录(例如命名为\code{scf}),将结构优化得到的\code{CONTCAR}文件复制(或移动)到\code{scf}目录内,并重命名为\code{POSCAR}。

将结构优化的\code{KPOINTS}和\code{POTCAR}复制(或移动)到\code{scf}目录内。

使用\code{vaspkit-101-ST}命令生成自洽计算所需要的\code{INCAR}文件。其中需要将截断能\code{ENCUT}设置为结构优化所使用的标准。修改后得到的\code{INCAR}文件如下所示:

\begin{lstlisting}[caption=INCAR]
Global Parameters
ISTART =  1
LREAL  = .FALSE.
ENCUT  =  600
PREC   =  Accurate
LWAVE  = .FALSE.
LCHARG = .TRUE. 
ADDGRID= .TRUE. 

Static Calculation
ISMEAR =  0
SIGMA  =  0.05
LORBIT =  11
NEDOS  =  2001
NELM   =  60
EDIFF  =  1E-08
\end{lstlisting}

其中需要特别注意的设置是:

\begin{itemize}
    \item \keyword{LWAVE}:表示写入波函数文件\code{WAVECAR},通常用于继续计算时的初始化设定。由于文件较大,因此如无必要,通常可以将其设定为\code{.FALSE.}表示不写入文件。
    \item \keyword{LCHARG}:表示写入电荷密度文件\code{CHGCAR}。在计算能带时,由于需要使用到这一文件,因此需要将其设定为\code{.TRUE.}。
    \item \keyword{EDIFF}:表示电子收敛标准。当能量变化达到这一标准时结束迭代计算。
\end{itemize}

将所有文件准备好后提交任务。

\subsection{能带计算}\label{subsec:VASP计算能带过程-能带计算}

相比于自洽计算,能带计算所需要的K点是特殊的\emph{高对称点路径},因此关键在于KPOINTS的生成。

将自洽计算得到的\code{CHGCAR}, \code{POSCAR}, \code{INCAR}, \code{POTCAR}全部复制(或移动)到一个新的目录下(假设为\code{band})。

使用\code{vaspkit-3}生成计算能带所用\code{KPOINTS}文件。其中需要根据结构特点选择是二维材料还是三维材料,在本例中由于\ch{SiO2}是二维材料,因此使用\code{KPOINTS-303}生成\code{KPATH.in}文件,将其命名为\code{KPOINTS}文件。生成后得到的文件如下所示。

\begin{lstlisting}[caption=KPOINTS]
K-Path Generated by VASPKIT.
   20
Line-Mode
Reciprocal
   0.0000000000   0.0000000000   0.0000000000     GAMMA          
   0.0000000000   0.0000000000   0.5000000000     X              
 
   0.0000000000   0.0000000000   0.5000000000     X              
   0.2500000000   0.2500000000   0.2500000000     P              
 
   0.2500000000   0.2500000000   0.2500000000     P              
   0.0000000000   0.5000000000   0.0000000000     N              
 
   0.0000000000   0.5000000000   0.0000000000     N              
   0.0000000000   0.0000000000   0.0000000000     GAMMA          
 
   0.0000000000   0.0000000000   0.0000000000     GAMMA          
   0.5000000000   0.5000000000  -0.5000000000     M              
 
   0.5000000000   0.5000000000  -0.5000000000     M              
   0.3674577537   0.6325422463  -0.3674577537     S              
 
  -0.3674577537   0.3674577537   0.3674577537     S_0            
   0.0000000000   0.0000000000   0.0000000000     GAMMA          
 
   0.0000000000   0.0000000000   0.5000000000     X              
  -0.2349155075   0.2349155075   0.5000000000     R              
 
   0.5000000000   0.5000000000  -0.2349155075     G              
   0.5000000000   0.5000000000  -0.5000000000     M              
\end{lstlisting}

其中每相邻两个点都是对应于一条高对称点路径。

对于\code{INCAR}文件,需要在自洽所使用文件的基础上,添加\code{ICHARG=11}表示\emph{读取当前目录下的\code{CHGCAR}文件},从而用于非自洽计算。

\begin{extend}
    在这里我们提到了\emph{自洽计算}和\emph{非自洽计算},简单来说,自洽计算就是用来计算电子结构最稳定状态,而非自洽计算则是利用这一结构计算电子的其他性质(如能带、态密度等)。
\end{extend}

将以上文件整理后提交任务计算。至此我们就已经完成了VASP计算能带的所有过程。


% \subsection{错误处理}\label{subsec:节标题-错误处理}
% % 请在本节列出可能遇见的错误与解决方法

% \subsubsection{错误1}

% \subsubsection{错误2}

% \subsubsection{错误3}