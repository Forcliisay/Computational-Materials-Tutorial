\section{具体计算步骤}\label{sec:具体计算步骤}

\sectionAuthor{Isay K.}

\begin{Abstract}
    \item 计算声子谱前的结构优化
    \item DFPT方法计算声子谱
    \item 有限位移法计算声子谱
\end{Abstract}

\subsection{计算声子谱前的结构优化}\label{sec:具体计算步骤-计算声子谱前的结构优化}

\begin{attention}
  在计算声子时需要先对原胞结构做高精度的结构优化,不然得到的声子谱中很容易出现虚频。  
\end{attention}

我们以$TiTe_2$为例,以下是高精度优化的具体参数。

\begin{lstlisting}[caption=INCAR]
Global Parameters
ISTART =  1            (Read existing wavefunction, if there)
ISPIN  =  1            (Non-Spin polarised DFT)
LREAL  = .FALSE.       (Projection operators: automatic)
ENCUT  =  380        (Cut-off energy for plane wave basis set, in eV)

LWAVE  = .FALSE.        (Write WAVECAR or not)
LCHARG = .FALSE.        (Write CHGCAR or not)
ADDGRID= .TRUE.        (Increase grid, helps GGA convergence)
LASPH  = .TRUE.        (Give more accurate total energies and band structure calculations)
PREC   = Accurate      (Accurate strictly avoids any aliasing or wrap around errors)
NCORE = 8 
ISYM = 0

Lattice Relaxation
NSW    =  300          (number of ionic steps)
ISMEAR =  0            (gaussian smearing method )
SIGMA  =  0.03         (please check the width of the smearing)
IBRION =  2            (Algorithm: 0-MD, 1-Quasi-New, 2-CG)
ISIF   =  3            (optimize atomic coordinates and lattice parameters)
IOPTCELL = 1 0 0 1 1 0 0 0 0
EDIFF = 1E-08
EDIFFG = -1E-03      (Ionic convergence, eV/A)
\end{lstlisting}

为了保证优化精度足够高,其中需要注意的是:
\begin{enumerate}
    \item EDIFF表示电子收敛标准,至少要取1E-06,体系小的话尽量取1E-08;
    \item EDIFFG取负值时表示力收敛标准,取1E-03;
    \item ADDGRID表示是否添加额外网格提高精度,设定为.TRUE.;
    \item PREC表示“精度”模式,设定为Accurate(准确);
    \item NSW表示电子优化步数,取300防止计算中断;
    \item ENCUT可以自行做测试,详见VASP计算-结构优化章节(先别去找,我没写);
\end{enumerate}

另外,其中ISIF=3表示既优化晶格又优化原子坐标,配合IOPTCELL可以实现晶轴的单独固定,以达到计算二维材料的目的,详见VASP计算-结构优化章节(也还没写)。

下面的其他输入文件没有需要特别说明的,如有疑问请参考VASP计算-结构优化章节(哈哈,又是这)或参考VASP官网:https://www.vasp.at/wiki/index.php/The\_VASP\_Manual

\begin{lstlisting}[caption=KPOINTS]
A
0
Gamma
24   24   1
0.0  0.0  0.0
\end{lstlisting}

\begin{lstlisting}[caption=POSCAR]
 TiTe2-1m1                               
    1.00000000000000     
      3.7458432095936396   -0.0000184453725456    0.0000000000000000
     -1.8729216047968198    3.2439861579338527    0.0000000000000000
      0.0000000000000000    0.0000000000000000   18.0000000000000000
    Ti   Te
      1     2
 Direct
   0.0000000000000000  0.0000000000000000  0.5127400160000022
   0.6666666132020026  0.3333334158033583  0.6098496477195335
   0.3333334157979960  0.6666666131966403  0.4156403382804701
  
   0.00000000E+00  0.00000000E+00  0.00000000E+00
   0.00000000E+00  0.00000000E+00  0.00000000E+00
   0.00000000E+00  0.00000000E+00  0.00000000E+00
\end{lstlisting}

\begin{lstlisting}[caption=POTCAR]
    (结构优化章节)
\end{lstlisting}

提交任务进行计算,得到CONTCAR为优化后的更合理的结构,作为后续声子计算的初始晶胞。
(后续小节中提到“初始晶胞”均指优化后得到的晶胞,为避免歧义在此说明。)

 
\subsection{DFPT方法计算声子谱}\label{sec:具体计算步骤-DFPT方法计算声子谱}

1.\code{mkdir method\_DFPT}

新建文件夹。

2.\code{cp relax/CONTCAR method\_DFPT/POSCAR}

将上一步高精度结构优化得到的CONTCAR复制进文件夹内,并重命名为POSCAR。

3.\code{cd method\_DFPT}

进入新文件夹。

4.\code{module load conda}
\code{conda activate phonopy}

加载conda模块,并激活phonopy环境,详情可参考\ref{sec:计算软件PHONOPY-PHONOPY使用方法}

5.\code{phonopy -d --dim="6 6 1"}

使用PHONOPY进行6×6的扩胞。

此时会产生数个名为POSCAR-0?的位移文件,以及名为SPOSCAR的扩胞后的结构。

DFPT方法使用的是SPOSCAR,而有限位移法使用的是这些位移文件。

\begin{attention}
   笔者研究的是2D结构,仅对两个方向进行扩胞,读者可根据需要自行调整。
   扩胞的标准是扩胞后达到80~100个原子,且晶轴长度大于10埃,不然得到的声子谱中很容易出现虚频。
\end{attention}

6.\code{mkdir vasp-calculations}

新建文件夹用于后续计算。

\begin{extend}
  此处根据个人习惯不同,也可以不新建文件夹。将POSCAR重命名为POSCAR-unit,将第5步新产生的SPOSCAR重命名为POSCAR,直接在当前文件夹中进行计算。
\end{extend}

7.\code{cp SPOSCAR vasp-calculations/POSCAR}

将第5步新产生的SPOSCAR复制进文件夹,并重命名为POSCAR。

8.准备其它基本文件:

\begin{lstlisting}[caption=INCAR]
SYSTEM = TiTe2
#ISIF = 3
NSW = 1
IBRION = 8

LWAVE = F
LCHARG = F

ENCUT = 380
EDIFF = 1E-8
EDIFFG =-1E-3
ISMEAR = 0

LREAL = F
SIGMA = 0.03

PREC = A
ADDGRID = .TRUE.
\end{lstlisting}

\begin{lstlisting}[caption=KPOINTS]
  A
  0
  Gamma
  3   3   1
  0.0  0.0  0.0
\end{lstlisting}

\begin{attention}
 因为该计算使用的是扩胞之后的结构,所以K点没有必要取太大。
\end{attention}

\begin{lstlisting}[caption=POTCAR]
  老方法
\end{lstlisting}


9.\code{sbatch sub.vasp}

提交任务进行计算。

10.\code{cd ..}

返回method\_DFPT文件夹。

11.\code{cp vasp-calculations/vasprun.xml .}

将vasprun.xml复制到当前文件夹。

12.\code{phonopy --fc vasprun.xml}

使用phonopy读取vasprun.xml生成力常数文件FORCE\_CONSTANTS。

13.\code{vi band.conf}

编辑band.conf文件:
\begin{lstlisting}[caption=band.conf]
ATOM_NAME =Ti Te
DIM = 6 6 1
BAND =0 0 0  0.5 0 0  0.33333 0.33333 0   0 0 0
BAND_POINTS = 101
FORCE_CONSTANTS = READ
\end{lstlisting}

\begin{enumerate}
    \item DIM根据体系的扩胞大小设置,如扩胞扩到332,就设置成332。
    \item BAND和能带的取点是一样的,也可以用vaspkit生成。
    \item FORCE\_CONSTANTS一定设置成READ。
    \item 更多设置可以看PHONOPY官网。
\end{enumerate}

14.\code{phonopy -p -s band.conf}

使用phonopy读取band.conf文件,作图并保存。

15.\code{phonopy-bandplot --gnuplot > phonon.out}

将数据导出方便后续用Origin等软件重新绘图。

\begin{extend}
  旧版本的phonopy的导出命令为:\code{bandplot  --gnuplot> phonon.out}
\end{extend}

\subsection{有限位移法计算声子谱}\label{sec:具体计算步骤-有限位移法计算声子谱}

1.\code{mkdir method\_yxwy}

2.\code{cp relax/CONTCAR method\_yxwy/POSCAR}

3.\code{cd method\_yxwy}

4.\code{phonopy -d --dim="6 6 1"}

前四步和DFPT法完全相同。

5.\code{for i in {01..12}; do mkdir \$i; cp POSCAR-0\$i \$i/POSCAR;done}

假如第四步产生了12个位移文件,使用for循环生成12个文件夹,并将对应的位移POSCAR移入文件夹重命名为POSCAR。

6.准备其它基本文件:

\begin{lstlisting}[caption=INCAR]
ADDGRID = .TRUE.
PREC = Accurate
IBRION = -1
ENCUT = 380
EDIFF = 1E-8
EDIFFG = -1E-3

ISMEAR = 0
SIGMA = 0.03

IALGO = 38

LREAL = .FALSE.
LWAVE = .FALSE.
LCHARG = .FALSE.

NCORE = 4
\end{lstlisting}

\begin{attention}
 有限位移法的单个计算实际上就是高精度的静态自洽。
\end{attention}

\begin{lstlisting}[caption=KPOINTS]
Automatic mesh
0
Gamma
3 3 1
0 0 0
\end{lstlisting}

\begin{attention}
 同DFPT法一样,有限位移法使用的也是扩胞之后的结构,所以K点没有必要取太大。
\end{attention}

7.\code{for i in {01..12}; do cp INCAR KPOINTS POTCAR sub.vasp \$i; cd \$i; sbatch sub.vasp; cd \$OLDPWD;done}

将基本文件复制进各个小文件夹中并进行计算。

8.\code{phonopy -f {01..12}/vasprun.xml}

计算全部结束后,使用phonopy读取全部的计算文件夹中的vasprun.xml,生成FORCE\_SETS文件。

9.\code{vi band.conf}

\begin{lstlisting}[caption=band.conf]
  ATOM_NAME =Ti Te
  DIM = 6 6 1
  BAND =0 0 0  0.5 0 0  0.33333 0.33333 0   0 0 0
  BAND_POINTS = 101
  FORCE_SETS = READ
\end{lstlisting}

与DFPT方法唯一不同的部分在于将FORCE\_CONSTANTS=READ改成FORCE\_SETS=READ。

10.\code{phonopy -p -s band.conf}

使用phonopy读取band.conf文件,作图并保存。

11.\code{phonopy-bandplot --gnuplot > phonon.out}

将数据导出方便后续用Origin等软件重新绘图。
