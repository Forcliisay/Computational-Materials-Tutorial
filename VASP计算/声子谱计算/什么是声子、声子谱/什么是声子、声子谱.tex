\section{什么是声子、声子谱} \label{sec:什么是声子、声子谱}

\sectionAuthor{Isay K.}

\begin{Abstract}
    \item 声子
    \item 声子谱
\end{Abstract}

\subsection{声子}\label{sec:什么是声子、声子谱-声子}

声子(Phonon),即“晶格振动的简正模能量量子”,是晶体中\emph{原子振动}的量子化描述。

在固体物理学中,声子是晶格振动的准粒子,其携带能量和动量,并且可以像粒子一样进行相互作用。

声子是\emph{简谐近似}下的产物,如果振动太剧烈,超过小振动的范围,那么晶格振动就要用非简谐振动理论描述。

声子并不是一个真正的粒子,声子可以产生和消灭,有相互作用的声子数不守恒,声子动量的守恒律也不同于一般的粒子,并且声子不能脱离固体存在。声子只是格波激发的量子,在多体理论中称为集体振荡的元激发或准粒子。

声子的化学势为零,属于\emph{玻色子},服从玻色-爱因斯坦统计。声子本身并不具有物理动量,但是携带有准动量,并具有能量,它的能量等于$\hbar\omega_q$。

声子可以分为以下两类:
\begin{itemize}
    \item 声学支:与晶格的纵向和横向振动相关,类似于声波,表示原胞的整体振动。
    \item 光学支:与晶格的非均匀振动相关,通常与电荷的重新分布有关,表示原胞内原子间的相互振动。
\end{itemize}

如果一个材料的原胞中有N个原子,那么声子谱就会有3N支,其中3条声学支,3N-3条光学支。

\subsection{声子谱}\label{sec:什么是声子、声子谱-声子谱}

声子谱,也称为声子色散关系,是描述声子能量与动量之间关系的图表。

声子谱通常在第一布里渊区内绘制,因为其包含了所有可能的声子模式。

通常,使用声子谱研究体系的动力学稳定性,使用分子动力学研究体系的热力学稳定性。

声子谱的其他物理意义:
\begin{itemize}
    \item 电子-声子耦合:在半导体和超导体中,电子-声子耦合相互作用对材料的电子性质至关重要;
    \item 声子散射:在金属和半导体中,声子散射是影响电子迁移率的关键因素;
    \item 热容:声子谱可以解释材料在不同温度下的热容行为
    \item ...
\end{itemize}
